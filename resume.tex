%% start of file `template-zh.tex'.
%% Copyright 2006-2013 Xavier Danaux (xdanaux@gmail.com).
%
% This work may be distributed and/or modified under the
% conditions of the LaTeX Project Public License version 1.3c,
% available at http://www.latex-project.org/lppl/.


\documentclass[11pt,a4paper,sans]{moderncv}   % possible options include font size ('10pt', '11pt' and '12pt'), paper size ('a4paper', 'letterpaper', 'a5paper', 'legalpaper', 'executivepaper' and 'landscape') and font family ('sans' and 'roman')

\moderncvstyle{classic}                        % options ‘casual’, ‘classic’, ‘oldstyle’ 和 ’banking’
\moderncvcolor{blue}                           % options ‘blue’ (默认)、‘orange’、‘green’、‘red’、‘purple’ 和 ‘grey’
%\nopagenumbers{}

\usepackage[utf8]{inputenc}
\usepackage{CJKutf8}

\usepackage[scale=0.75]{geometry}
\setlength{\hintscolumnwidth}{3cm}           

\name{高远}{}
\title{justgaoyuan}
\phone[mobile]{15951989141}
\email{justgaoyuan@gmail.com}
\address{南京浦口区}
\homepage{https://github.com/justgaoyuan}
\extrainfo{}
\photo[64pt][0.1pt]{picture}
%\quote{}

\makeatletter
\renewcommand*{\bibliographyitemlabel}{\@biblabel{\arabic{enumiv}}}
\makeatother

\begin{document}
\begin{CJK}{UTF8}{gbsn}
\maketitle

\section{基本信息}
\cvcomputer{姓名}{高远}{性别}{男}
\cvcomputer{民族}{汉族}{籍贯}{江苏宿迁}
\cvcomputer{学历}{本科}{专业}{计算机科学与技术}
\cvcomputer{出生日期}{1989-4-20}{政治面貌}{中共党员}

\section{工作背景}
\cventry{2012.5 -- 至今}{嵌入式软件工程师}{南京}{吉隆光纤通信股份有限公司}{}{
\begin{itemize}%
	\item 熔接机软件:驱动、业务、图像处理、校正算法等;
	\item 开发环境主要是Arch Linux , 主要使用C/C++, 脚本语言bash也有一定的使用经验;
	\item 熟悉openembeded、yocto,使用其构建嵌入式linux系统;
	\item 软件版本管理,小团队代码托管服务器的维护,软件包编译及发布;
	\item gitweb / gitlab / ftp / tftp / http / samba / iptables 有一定的配置维护经验;
	\item 阅读同类产品厂商专利,理解并进行尝试。
\end{itemize}}

\section{项目经验}
\cventry{}{光纤熔接机控制软件开发}{}{}{}{
\begin{itemize}%
	\item 项目描述:
		\begin{itemize}
			\item 光纤推进与对准:通过两路cmos镜头获取图像,对图像进行分析,推进和对准光纤;
			\item 放电:电极放电,电弧会产生高温,将已经对准的两条光纤的前端融化,并同时推进两根光纤将两根光纤熔接起来。
		\end{itemize}
	\item 责任描述
		\begin{itemize}
			\item 温度、湿度、气压传感器,数字电位器等驱动程序开发;
			\item 封装设备(步进电机,加热器,温度气压湿度传感器,高压板,摄像头等)为上层应用提供接口;
			\item 具体业务功能模块开发;
			\item 图像处理算法,放电强度校正,实时校正算法等。
		\end{itemize}
\end{itemize}}
\cventry{}{高速摄像板软件开发}{}{}{}{
	\begin{itemize}
	\item 项目描述
		\begin{itemize}
			\item ARM(dm3730)+FPGA+SENSOR(vita1300)
			\item该板主要针对光纤熔接机的熔接过程而设计,采用高速的CMOS芯片能够拍到光纤的熔解过程,一秒能达到500帧的速率,使用DDR芯片存储拍到的图像数据,能够存最少10S的图像数据,ARM将存储的图像数据取出在显示器上显示出来,ARM和FPGA之间采用GPMC接口,FPGA实现和DDR芯片以及CMOS芯片之间接口,FPGA和CMOS芯片之间是并行高速接口,并将像素点从帧中解出存入DDR芯片。
		\end{itemize}
	\item 责任描述
		\begin{itemize}
			\item 配合硬件测试硬件模块的基本功能usb、网口、SD卡、VGA等。
			\item vita1300 调试及驱动程序。
			\item Arm从DDR从读数据并保存到SD卡上,图像数据的慢速回放及存储。
		\end{itemize}
	\end{itemize}}

\section{语言技能}
\cvitemwithcomment{英语}{四级}{}

% \section{计算机技能}
% \cvdoubleitem{类别 1}{XXX, YYY, ZZZ}{类别 4}{XXX, YYY, ZZZ}

\section{个人兴趣}
\cvitem{篮球}{\small 控球后卫,喜欢团队篮球,享受彼此配合共同努力争取胜利。}

\renewcommand{\listitemsymbol}{-}

\section{自我评价}
\cvline{}{勤奋严谨,有钻研精神,动手能力强,有很强的自学能力}

\section{教育背景}
\cventry{2008--2012}{本科}{江苏科技大学}{镇江}{\textit{计算机科学与技术}}{}


\nocite{*}
\bibliographystyle{plain}
\bibliography{publications}

\clearpage\end{CJK}
\end{document}
